%!TEX root = ./main.tex

\section{Introduction}\label{sec:intro}
The gravity field of a giant planet is perhaps our best window into its interior
structure and composition. Obtaining gravity data is no simple task. Rough
estimates can be deduced form ground-based observations of natural satellites,
but precise measurements come from tracking radio signals from probes sent to
the outer solar system to orbit, or at least fly by these worlds. Turning the
hard-won gravity data into predictions about a planet's interior is also not a
straightforward task. There is no simple inversion; the usual process involves
creating models of the planet's interior and making claims about features of the
planet we are interested in, for example the existence and size of a solid core,
based on how well the calculated gravity of model planets with such features
matches the observed gravity if the real planet. What may not be well
appreciated outside the subset of researchers who routinely work with these
interior models are the host of assumptions that go into them, making virtually
every inference about a giant planet's structure strongly model dependent.

Physical interior models try to create a self-consistent description of the
composition, density, pressure, and temperature, at every point inside the
planet. Gravity comes in with the requirement of hydrostatic equilibrium,
connecting the pressure $P$ (actually its gradient) and density $\rho$ at every
point. In a self-consistent model these quantities are also related via a
thermodynamic equation of state (EOS) which also requires knowledge, at every
point, of temperature and composition. The temperature can be calculated
self consistently following equations of heat transfer, leading to
time-dependent colling models, or a static temperature (or, equivalently,
entropy) structure can be pre-calculated and used throughout. Composition,
however, cannot be calculated; it must be stipulated.

And therein lies the rub: the model must assume the very thing it is supposed to
infer. To be sure, there are some very good assumptions that one can make. For
example, if the target planet is a gas giant we may assume that, in a large
fraction of the volume of the interior the dominant species is a mix of hydrogen
and helium. But in what proportion? This should be a model parameter, call it
$Y$; more precisely $Y(r)$ because in principle it should be allowed to change
with distance from the center, $r$. In the case of the ice giants, on the other
hand, the dominant species is not so easy to guess. And of course, the dominant
species is not the only species. Some of the most important questions about
planet formation that are perhaps the main motivation for the model in the first
place depend most strongly on the inferred content of heavier elements. Make it
a model parameter, $Z(r)$, under the (sometimes unspoken) assumption that the
outcome is not overly sensitive to exactly \emph{which} heavy element we will
use in the model.

In fact, this hypothetical model is already much too complex. To resolve the
interior to a meaningful degree requires discretizing the continuous variables
on a fine grid in $r$, with at least hundreds and preferably thousands of grid
points, resulting in thousands of model parameters and an
impracticable\footnote{Or is it? It may be the brute computational force is up
to the challenge. But this is an idea for another time.} task. Some very strong
simplifying assumptions are need. The most important one is the assumption of
some kind of layering.


\begin{figure}[tb!]
\centering
\plotone{N13_profiles}
\caption{Density profiles of the three-layer models of \citet{Nettelmann2013b},
replotted from their data.}
\label{fig:N13_profs}
\end{figure}
