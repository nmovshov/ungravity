%!TEX root = ./main.tex

\section{A methods section}\label{sec:method}
\subsection{Parameterization}\label{sec:params}

Consider the density profiles of the three-layer Uranus and Neptune models of
\citet{Nettelmann2013b}, shown in Figure~\ref{fig:N13_profs}. These profiles are
derived from models that solve the planetary structure equations for an assumed
composition, using physical equations of state. Our goal is to find a suitable
direct parameterization of density as a function of radius, $\rho(s)$, that is
capable of capturing the profiles of Fig.~\ref{fig:N13_profs} as a subset and is
otherwise as flexible as possible. It is also important to keep the number of
parameters small and, even more important, to minimize parameter correlations.
The more parameters we use and the stronger the correlations between them, the
longer it will take any sampling algorithm to adequately cover the sample space.

With these requirements in mind, what are the important features of the curves
in Fig.~\ref{fig:N13_profs} that we need to allow for? Each $\rho(s)$ curve is
smooth and monotonic, except for two sharp discontinuities. In the physical
models these discontinuities are ``built-in''; the models assume layers of
homogeneous composition with sharp boundaries between them. Three-layer models
thus have two density discontinuities, the implicit assumption being that
regions of composition gradients of length scale below the model's resolution
separate the layers. The smoothness of the $\rho(s)$ curve between these density
``jumps'' suggests that the entire profile can be well-represented by a
piecewise-polynomial function. Such a parameterization was used successfully in
\citep{Movshovitz2020} to generate density profiles for Saturn, but here we
utilize an alternative parameterization that is superior to the
piecewise-polynomial in two significant ways, both having to do with
representing the density jumps.

There are two important limitations to representing $\roofs$ with
piecewise-polynomials. The first and more obvious is that this parameterization
only allows for sharp density discontinuities, not gradual ones. There are two
mathematical discontinuities (of the first kind) in every density profile, with
parameters controlling their location, say fractions $z_1$ and $z_2$ of the
planet's radius, and jump magnitude, say $\delta_1$ and $\delta_2$ in density
units\footnote{For reasons of efficiency one may choose to use some one-to-one
transformation of these parameters \citep[e.g.][Appendix B]{Movshovitz2020}, but
their physical meaning remains.}. The jumps can merge ($z_1\to{}z_2$) or one or
both may vanish ($\delta_i\to{0}$), but they cannot approximate a more gradual
transition, a gradient region detectable on the scale of the model. These sharp
density jumps are features of layered composition models and we want our
parameterization to be able to reproduce them, but we would like it to have the
flexibility to capture gradient regions as well, something that
composition-based models have difficulty with.

The second limitation of the piecewise-polynomial parameterization is a
technical one, that becomes apparent when we try to use MCMC algorithms to
sample from the parameters' joint posterior. It turns out that the parameters
are very strongly correlated leading to very low mixing rate and impractically
long convergence time. One way to mitigate this problem is to fix values of the
parameters $z_1$ and $z_2$, obtain their marginal distributions by sampling the
other parameter values under their conditional probability, and repeat the
process for a range of reasonable values for $z_i$. A sample from the full
posterior can be created by drawing from the marginal probabilities, in
proportion to their relative likelihoods. While this approach provides a working
method, it is more cumbersome and time consuming. Worse, the additional step of
combining the marginals into a single posterior involves the difficult task of
calculating, at least approximately, the \emph{posterior odds} ratio (also
called the Bayes factor or the evidence integral or, simply, the evidence).
While this is a common and well-studied task, it still has no generally agreed
upon best method. It would be better if we didn't have to go through this step.

Our alternative parameterization represents $\roofs$ with a single continuous
and continuously differentiable function:
\begin{multline}\label{eq:ppwd}
\roofz = \rho(s/\rem) = \sum_{n=2}^{8}a_n(z^n - 1) + \rho_0 + \\
\sum_{n=1}^{2}\frac{\sigma_n}{\pi}\Bigl(\frac{\pi}{2} +
\arctan\bigl(-\nu_n(z - z_n)\bigr)\Bigr).
\end{multline}
The first line is a degree-8 polynomial in $z=s/\rem$. It is constrained to have
a vanishing derivative at $z=0$ and to pass through the point $(1,\rho_0)$.
Since we reference our $J$ values to the 1-bar surface we take
$\rho_0=\rho\sub{1bar}$ to be the 1-bar density. For Uranus
$\rho\sub{1bar,U}=0.367\unit{kg/m^3}$ and for Neptune
$\rho\sub{1bar,N}=0.387\unit{kg/m^3}$, but it is important to remember that
these nominal values do not come from direct measurements. They are derived by
applying the ideal gas law to a protosolar mix of hydrogen and helium at a
temperature $T\sub{1bar}$, which itself is inferred from analysis of radio
occultation data from the Voyager 2 mission \citep{Lindal1992}. \edit1{Jonathan
did I get that right?} In section~\ref{sec:results} we compare posterior
distributions of density profiles obtained by taking $\rho_0$ to be constant
with those obtained by treating it as a regular sampled parameter.

The second line in eq.~\eqref{eq:ppwd} is a parameterization of two potential
density jumps, overlain on top of the polynomial. The parameter $z_i$ define the
location (in normalized radius) of the center of the ``jump''. A density
increase of $\sigma_i$ (in density units) is applied asymptotically around this
point, the width being controlled by the non-dimensional sharpness parameter
$\nu_i$. The jump can be made arbitrarily sharp to resemble a discontinuity like
the ones in Fig.~\ref{fig:N13_profs} by increasing the value of $\nu_i$.
Conversely, small values of $\nu_i$ result in smooth, gradual density increase,
indistinguishable from the background polynomial.

The number of parameters required to completely define the density profile is 13
(if we take $\rho_0$ to be constant). A degree-8 polynomial with two boundary
conditions takes 7 parameters, and two potential density jumps take three
parameters each (location, scale, and sharpness). This is comparable to the
piecewise-polynomial parameterization. The advantage is not in the number of
parameters but in their correlations. The polynomial-plus-atan parameters are
less strongly correlated, meaning a small change in one parameter while keeping
the other parameters fixed results in a smaller overall change in the density
profile. This significantly improves the behavior of the sampling algorithm,
which nonetheless remains a computationally expensive operation. Equally
important, this parameterization is significantly more flexible in treatment of
possible density jumps, allowing for one or both to be either sharp or gradual.