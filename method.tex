%!TEX root = ./main.tex

\section{A methods section}\label{sec:method}
\subsection{Parameterization}\label{sec:params}

Consider the density profiles of the three-layer Uranus and Neptune models of
\citet{Nettelmann2013b}, shown in Figure~\ref{fig:N13_profs}. These profiles are
derived from models that solve the planetary structure equations for an assumed
composition, using physical equations of state. Our goal is to find a suitable
direct parameterization of density as a function of radius, $\rho(s)$, that is
capable of capturing the profiles of Fig.~\ref{fig:N13_profs} as a subset and is
otherwise as flexible as possible. It is also important to keep the number of
parameters small and, even more important, to minimize parameter correlations.
The more parameters we use and the stronger the correlations between them, the
longer it will take any sampling algorithm to adequately cover the sample space.

With these requirements in mind, what are the important features of the curves
in Fig.~\ref{fig:N13_profs} that we need to allow for? Each $\rho(s)$ curve is
smooth and monotonic, except for two sharp discontinuities. In the physical
models these discontinuities are ``built-in''; the models assume layers of
homogeneous composition with sharp boundaries between them. Three-layer models
thus have two density discontinuities, the implicit assumption being that
regions of composition gradients of length scale below the model's resolution
separate the layers. The smoothness of the $\rho(s)$ curve between these density
``jumps'' suggests that the entire profile can be well-represented by a
piecewise-polynomial function. Such a parameterization was used successfully in
\citep{Movshovitz2020} to generate density profiles for Saturn, but here we
utilize an alternative parameterization that is superior to the
piecewise-polynomial in two significant ways, both having to do with
representing the density jumps.

There are two important limitations to representing $\roofs$ with
piecewise-polynomials. The first and more obvious is that this parameterization
only allows for sharp density discontinuities, not gradual ones. There are two
mathematical discontinuities (of the first kind) in every density profile, with
parameters controlling their location, say fractions $z_1$ and $z_2$ of the
planet's radius, and jump magnitude, say $\delta_1$ and $\delta_2$ in density
units\footnote{For reasons of efficiency one may choose to use some one-to-one
transformation of these parameters \citep[e.g.][Appendix B]{Movshovitz2020}, but
their physical meaning remains.}. The jumps can merge ($z_1\to{}z_2$) or one or
both may vanish ($\delta_i\to{0}$), but they cannot approximate a more gradual
transition, a gradient region detectable on the scale of the model. These sharp
density jumps are features of layered composition models and we want our
parameterization to be able to reproduce them, but we would like it to have the
flexibility to capture gradient regions as well, something that
composition-based models have difficulty with.

The second limitation of the piecewise-polynomial parameterization is a
technical one, that becomes apparent when we try to use MCMC algorithms to
sample from the parameters' joint posterior. It turns out that the parameters
are very strongly correlated leading to very low mixing rate and impractically
long convergence time. One way to mitigate this problem is to fix values of the
parameters $z_1$ and $z_2$, obtain their marginal distributions by sampling the
other parameter values under their conditional probability, and repeat the
process for a range of reasonable values for $z_i$. A sample from the full
posterior can be created by drawing from the marginal probabilities, in
proportion to their relative likelihoods. While this approach provides a working
method, it is more cumbersome and time consuming. Worse, the additional step of
combining the marginals into a single posterior involves the difficult task of
calculating, at least approximately, the \emph{posterior odds} ratio (also
called the Bayes factor or the evidence integral or, simply, the evidence).
While this is a common and well-studied task, it still has no generally agreed
upon best method. It would be better if we didn't have to go through this step.