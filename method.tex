%!TEX root = ./main.tex

\section{A methods section}\label{sec:method}
\subsection{Parameterization}\label{sec:params}

Consider the density profiles of the three-layer Uranus and Neptune models of
\citet{Nettelmann2013b}, shown in Figure~\ref{fig:N13_profs}. These profiles are
derived from models that solve the planetary structure equations for an assumed
composition, using physical equations of state. Our goal is to find a suitable
direct parameterization of density as a function of radius, $\rho(s)$, that is
capable of capturing the profiles of Fig.~\ref{fig:N13_profs} as a subset and is
otherwise as flexible as possible. It is also important to keep the number of
parameters small and, even more important, to minimize parameter correlations.
The more parameters we use and the stronger the correlations between them, the
longer it will take any sampling algorithm to adequately cover the sample space.

With these requirements in mind, what are the important features of the curves
in Fig.~\ref{fig:N13_profs} that we need to allow for? Each $\rho(s)$ curve is
smooth and monotonic, except for two sharp discontinuities. In the physical
models these discontinuities are ``built-in''; the models assume layers of
homogeneous composition with sharp boundaries between them. Three-layer models
thus have two density discontinuities, the implicit assumption being that
regions of composition gradients of length scale below the model's resolution
separate the layers. The smoothness of the $\rho(s)$ curve between these density
``jumps'' suggests that the entire profile can be well-represented by a
piecewise-polynomial function. Such a parameterization was used successfully in
\citep{Movshovitz2020} to generate density profiles for Saturn, but here we
would like to improve on the parameterization for two reasons.