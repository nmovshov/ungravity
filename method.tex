%!TEX root = ./main.tex

\section{Experimental method}\label{sec:method}

\subsection{Overview}\label{sec:overview}
In order to evaluate the potential of precise measurements of the gravity fields
to improve our knowledge of the interiors of the ice giants we generate samples
of interior density profiles that are constrained in different ways. By
comparing the range of density values attained by each sample we get an idea of
the ``constraining power'' of the observable quantity that guided the sampling.

Concretely, we start with a sample of density profiles with no knowledge of the
gravity field at all, to use as a baseline for comparison. Even without gravity,
the space of allowed $\roofs$ curves is already strongly constrained by basic
physics. Trivially, $\roofs$ must be monotonic strictly decreasing with $s$ and
its integral between $s=0$ and $s=\rem$ should match the planet's mass within
observable uncertainty. The relationship between pressure and density in
hydrostatic equilibrium also implies $\lim_{s\to{0+}}d\rho/ds=0$. By themselves
these conditions only restrict the shape of the $\roofs$ curve, not its scale,
but we can constrain the by putting reasonable limits on the density at both the
surface and the center of the planet.

On the surface, which in our models we take to mean the 1-bar surface, a density
can be estimated, indirectly, by applying the ideal gas law to a protosolar mix
of hydrogen and helium at the observed temperature, $T\sub{1bar}$. This leads to
a nominal value of $\rho\sub{1bar,U}=0.367\unit{kg/m^3}$ for Uranus, and for
Neptune $\rho\sub{1bar,N}=0.387\unit{kg/m^3}$. The 1bar-temperature itself is
not a direct measurement. It is inferred from analysis of radio occultation data
from the Voyager 2 mission \citep{Lindal1992}, and includes uncertainty of a few
degrees in each direction. \note{Jonathan did I get that right?} Combining the
uncertainty from temperature and composition, we should allow some ``play'' in
the 1-bar surface density, rather than treat it as a strict boundary condition.
We can do this easily by including it as a sampled variable with an appropriate
prior (appendix~\ref{sec:chains}).

The density at the center of the planet, of course, is not directly available.
But we can set a reasonable upper bound value by considering the order of
magnitude of the central pressure. We set the upper limit for both planets at
$20,000\unit{kg/m^3}$, comfortably above the density of pure rock at
${50}\unit{Mbar}$. The values used for these baseline models are summarized in
Table~\ref{tab:observables}.

%!TEX root = ./main.tex

\begin{deluxetable}{cccccc}\label{tab:observables}

%% Keep a portrait orientation

%% Over-ride the default font size
%% Use Default (12pt)

%% Use \tablewidth{?pt} to over-ride the default table width.
%% If you are unhappy with the default look at the end of the
%% *.log file to see what the default was set at before adjusting
%% this value.

%% This is the title of the table.
\tablecaption{Table 1}

%% This command over-rides LaTeX's natural table count
%% and replaces it with this number.  LaTeX will increment 
%% all other tables after this table based on this number
\tablenum{1}

%% The \tablehead gives provides the column headers.  It
%% is currently set up so that the column labels are on the
%% top line and the units surrounded by ()s are in the 
%% bottom line.  You may add more header information by writing
%% another line between these lines. For each column that requries
%% extra information be sure to include a \colhead{text} command
%% and remember to end any extra lines with \\ and include the 
%% correct number of &s.
\tablehead{\colhead{col1} & \colhead{col2} & \colhead{col3} & \colhead{col4} & \colhead{col5} & \colhead{col6} \\ 
\colhead{(m)} & \colhead{(kg)} & \colhead{(s)} & \colhead{(A)} & \colhead{(Mol)} & \colhead{(K)} } 

%% All data must appear between the \startdata and \enddata commands
\startdata
2444963.767  &   6.32\tablenotemark{d}  & 0.05  &  0.31  & 0.01  & JPA  \\ 
2444964.787  &   6.43                   & 0.02  &  0.22  & 0.02  & GJS  \\ 
2444965.682  &   6.66                   & 0.07  &  0.18  & 0.03  & KVan \\ 
2444969.590  &   7.00                   & 0.10  &  0.05  & 0.01  & GJS  \\ 
2444970.613  &   7.05\tablenotemark{d}  & 0.05  &  0.03  & 0.03  & JPA  \\ 
2444972.430  &   7.21\tablenotemark{d}  & 0.05  & -0.02  & 0.03  & GJS  \\ 
2444978.490  &   8.30                   & 0.08  & -0.11  & 0.02  & KVan \\ 
2445628.601  & <15.0                    & 0.10  &        &       & GJS \\ 
\enddata

%% Include any \tablenotetext{key}{text}, \tablerefs{ref list},
%% or \tablecomments{text} between the \enddata and 
%% \end{deluxetable} commands

%% No \tablecomments indicated

%% No \tablerefs indicated

\end{deluxetable}

\subsection{Parameterization of the density profiles}\label{sec:params}

Consider the density profiles of the three-layer Uranus and Neptune models of
\citet{Nettelmann2013b}, shown in Figure~\ref{fig:N13_profs}. These profiles are
derived from models that solve the planetary structure equations for an assumed
composition, using physical equations of state. Our goal is to find a suitable
direct parameterization of density as a function of radius, $\rho(s)$, that is
capable of capturing the profiles of Fig.~\ref{fig:N13_profs} as a subset and is
otherwise as flexible as possible. It is also important to keep the number of
parameters small and, even more important, to minimize parameter correlations.
The more parameters we use and the stronger the correlations between them, the
longer it will take any sampling algorithm to adequately cover the sample space.

With these requirements in mind, what are the important features of the curves
in Fig.~\ref{fig:N13_profs} that we need to allow for? Each $\rho(s)$ curve is
smooth and monotonic, except for two sharp discontinuities. In the physical
models these discontinuities are ``built-in''; the models assume layers of
homogeneous composition with sharp boundaries between them. Three-layer models
thus have two density discontinuities, the implicit assumption being that
regions of composition gradients of length scale below the model's resolution
separate the layers. The smoothness of the $\rho(s)$ curve between these density
``jumps'' suggests that the entire profile can be well-represented by a
piecewise-polynomial function. Such a parameterization was used successfully in
\citep{Movshovitz2020} to generate density profiles for Saturn, but here we
utilize an alternative parameterization that is superior to the
piecewise-polynomial in two significant ways, both having to do with
representing the density jumps.

There are two important limitations to representing $\roofs$ with
piecewise-polynomials. The first and more obvious is that this parameterization
only allows for sharp density discontinuities, not gradual ones. There are two
mathematical discontinuities (of the first kind) in every density profile, with
parameters controlling their location, say fractions $z_1$ and $z_2$ of the
planet's radius, and jump magnitude, say $\delta_1$ and $\delta_2$ in density
units\footnote{For reasons of efficiency one may choose to use some one-to-one
transformation of these parameters \citep[e.g.][Appendix B]{Movshovitz2020}, but
their physical meaning remains.}. The jumps can merge ($z_1\to{}z_2$) or one or
both may vanish ($\delta_i\to{0}$), but they cannot approximate a more gradual
transition, a gradient region detectable on the scale of the model. These sharp
density jumps are features of layered composition models and we want our
parameterization to be able to reproduce them, but we would like it to have the
flexibility to capture gradient regions as well, something that
composition-based models have difficulty with.

The second limitation of the piecewise-polynomial parameterization is a
technical one, that becomes apparent when we try to use MCMC algorithms to
sample from the parameters' joint posterior. It turns out that the parameters
are very strongly correlated leading to very low mixing rate and impractically
long convergence time. One way to mitigate this problem is to fix values of the
parameters $z_1$ and $z_2$, obtain their marginal distributions by sampling the
other parameter values under their conditional probability, and repeat the
process for a range of reasonable values for $z_i$. A sample from the full
posterior can be created by drawing from the marginal probabilities, in
proportion to their relative likelihoods. While this approach provides a working
method, it is more cumbersome and time consuming. Worse, the additional step of
combining the marginals into a single posterior involves the difficult task of
calculating, at least approximately, the \emph{posterior odds} ratio (also
called the Bayes factor or the evidence integral or, simply, the evidence).
While this is a common and well-studied task, it still has no generally agreed
upon best method. It would be better if we didn't have to go through this step.

Our alternative parameterization represents $\roofs$ with a single continuous
and continuously differentiable function:
\begin{multline}\label{eq:ppwd}
\roofz = \rho(s/\rem) = \sum_{n=2}^{8}a_n(z^n - 1) + \rho_0 + \\
\sum_{n=1}^{2}\frac{\sigma_n}{\pi}\Bigl(\frac{\pi}{2} +
\arctan\bigl(-\nu_n(z - z_n)\bigr)\Bigr).
\end{multline}
The first line is a degree-8 polynomial in $z=s/\rem$. It is constrained to have
a vanishing derivative at $z=0$ and to pass through the point $(1,\rho_0)$.
Since we reference our $J$ values to the 1-bar surface we take
$\rho_0=\rho\sub{1bar}$ to be the 1-bar density. For Uranus
$\rho\sub{1bar,U}=0.367\unit{kg/m^3}$ and for Neptune
$\rho\sub{1bar,N}=0.387\unit{kg/m^3}$, but it is important to remember that
these nominal values do not come from direct measurements. They are derived by
applying the ideal gas law to a protosolar mix of hydrogen and helium at a
temperature $T\sub{1bar}$, which itself is inferred from analysis of radio
occultation data from the Voyager 2 mission \citep{Lindal1992}. \note{Jonathan
did I get that right?} In section~\ref{sec:results} we compare posterior
distributions of density profiles obtained by taking $\rho_0$ to be constant
with those obtained by treating it as a regular sampled parameter.

The second line in eq.~\eqref{eq:ppwd} is a parameterization of possible density
jumps, overlain on top of the polynomial. In this version of the
parameterization we allow up to two such jumps. The parameter $z_1$ defines the
location (in normalized radius) of the center of the inner one. A density
increase of $\sigma_1$ (in density units) is applied asymptotically around this
point, the width being controlled by the non-dimensional sharpness parameter
$\nu_1$. The jump can be made arbitrarily sharp to resemble a discontinuity like
the ones in Fig.~\ref{fig:N13_profs} by increasing the value of $\nu_1$.
Conversely, small values of $\nu_1$ result in smooth, gradual density increase,
indistinguishable from the background polynomial. The location, scale, and
sharpness of the outer density jump are similarly determined by $z_2$,
$\sigma_2$, and $\nu_2$, respectively.

As given in eq.~\eqref{eq:ppwd} the number of parameters required to completely
define the density profile is 13 (if we take $\rho_0$ to be constant). A
degree-8 polynomial with two boundary conditions takes 7 parameters, and two
potential density jumps take three parameters each (location, scale, and
sharpness). Of course this polynomial-plus-arctan parameterization can be made
even more flexible, by increasing the polynomial degree (we explain in
appendix~\ref{sec:polyplanets} why we use a degree-8 polynomials) and/or adding
more arctan terms. But the advantages of added flexibility must be carefully
weighted against the cost of increasing the dimensionality of the sample space.

As it is, sampling from the 13-dimensional, polynomial-plus-atan parameter space
is computationally expensive operation, although much less so than with the
piecewise-polynomial parameterization. The advantage comes not from the number
of parameters, which is similar in both, but from the weaker parameter
correlations. The polynomial-plus-atan parameters are less strongly correlated,
meaning a small change in one parameter while keeping the other parameters fixed
results in a smaller overall change in the density profile. This significantly
improves the behavior of the sampling algorithm.

\subsection{Sampling procedure}\label{sec:mcmc}
To draw a sample of $\roofs$ profiles for each planet we run the ensemble
sampler of the \code{emcee} Python package \citep{Foreman-Mackey2013} which
implements the parallel stretch-move algorithm of \citet{Goodman2010}.